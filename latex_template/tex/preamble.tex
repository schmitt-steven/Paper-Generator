% Normally nothing needs to be adjusted here
\usepackage[pdftex]{graphicx}
\graphicspath{{images/}}
\DeclareGraphicsExtensions{.pdf,.jpeg,.jpg,.png}

% Modern academic font - Libertinus (elegant, readable, professional)
\usepackage{libertinus}
\usepackage[libertine]{newtxmath}  % This loads amsmath automatically
\usepackage{algorithmic}
\usepackage{array}
\usepackage{dblfloatfix}
\usepackage{url}
\usepackage[autostyle=true]{csquotes}
\usepackage[backend=biber,
            sorting=none,   % No sorting
            doi=true,       % Show DOI
            isbn=false,     % Don't show ISBN
            url=true,       % Show URLs
            maxnames=6,     % Use et al. from 6 authors
            minnames=1,     % and only show first author
            style=ieee,]{biblatex}
\usepackage{booktabs}
\usepackage{xcolor}
\usepackage{listings}             % Source Code listings
\usepackage[printonlyused]{acronym}
\usepackage{fancyvrb}
\usepackage{tocloft} % Better table of contents

% Define colors - Modern sleek palette
\definecolor{linkblue}{RGB}{41, 98, 255}        % Modern vibrant blue for links
\definecolor{linkblack}{RGB}{45, 45, 45}        % Soft black (not pure black)
\definecolor{darkgreen}{RGB}{16, 185, 129}      % Modern teal-green (emerald)
\definecolor{darkblue}{RGB}{59, 130, 246}       % Clean sky blue
\definecolor{darkred}{RGB}{239, 68, 68}         % Modern red (not too dark)
\definecolor{comment}{RGB}{107, 114, 128}       % Sophisticated gray for comments
\definecolor{javadoccomment}{RGB}{96, 165, 250} % Light blue for doc comments
\definecolor{keyword}{RGB}{168, 85, 247}        % Modern purple for keywords
\definecolor{type}{RGB}{34, 197, 94}            % Fresh green for types
\definecolor{method}{RGB}{249, 115, 22}         % Warm orange for methods
\definecolor{variable}{RGB}{45, 45, 45}         % Soft black for variables
\definecolor{literal}{RGB}{236, 72, 153}        % Modern pink/magenta for literals
\definecolor{operator}{RGB}{100, 116, 139}      % Slate gray for operators

\usepackage[english]{babel}

\DefineBibliographyStrings{english}{
    andothers = {{et al\adddot}},  % Always use et al.
}
\usepackage[
      unicode=true,
      hypertexnames=false,
      colorlinks=true,
      colorlinks=false,
      linkcolor=darkblue,
      citecolor=darkblue,
      urlcolor=darkblue,
      pdftex
   ]{hyperref}

% Settings for source code listings
\lstset{
    xleftmargin=0.1cm,
    basicstyle=\scriptsize\ttfamily,
    keywordstyle=\color{keyword},
    identifierstyle=\color{variable},
    commentstyle=\color{comment},
    stringstyle=\color{literal},
    tabsize=2,
    lineskip={2pt},
    columns=flexible,
    inputencoding=utf8,
    captionpos=b,
    breakautoindent=true,
    breakindent=2em,
    breaklines=true,
    prebreak=,
    postbreak=,
    numbers=none,
    numberstyle=\tiny,
    showspaces=false,      % No space symbols
    showtabs=false,        % No tab symbols
    showstringspaces=false,% Spaces in strings
    morecomment=[s][\color{javadoccomment}]{/**}{*/},
}

\hypersetup{
    pdftitle={\dokumententitel},
    pdfdisplaydoctitle=true,
    hidelinks
}

% Macros for typographically correct abbreviations
\newcommand{\eg}[0]{e.\,g.}
\newcommand{\ie}[0]{i.\,e.}
\newcommand{\etal}[0]{et.\,al.}

% Where is source code located?
\newcommand{\srcloc}{src/}

% Include bibliography
\addbibresource{literature.bib}
